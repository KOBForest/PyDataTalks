\documentclass[Master.tex]{subfiles}
\begin{document}
%=================================================%
\begin{frame}
\begin{figure}
\centering
\includegraphics[width=1.05\linewidth]{images/Julia-Logo-HypothesisTests}

\end{figure}

\end{frame}	
%============================================================%
\begin{frame}[fragile]
	\frametitle{Statistics with Julia}
	\large
	\begin{itemize}
		\item It’s all very well generating myriad statistics characterising your data. 
		\item How do you know whether or not those statistics are telling you something interesting? Hypothesis Tests. 
		\item To that end, we’ll be looking at the HypothesisTests package today.
	\end{itemize}

\end{frame}
%=================================================%
\begin{frame}[fragile]
	\frametitle{Statistics with Julia}
	\large
	\begin{verbatim}
Julia> using HypothesisTests
julia> using Distributions

julia> srand(357)
julia> x1 = rand(Normal(), 1000);
julia> x2 = rand(Normal(0.5, 1), 1000);

julia> # 25% success rate on samples of size 100
julia> x3 = rand(Binomial(100, 0.25), 1000); 

julia> # 50% success rate on samples of size 50  
julia> x4 = rand(Binomial(50, 0.50), 1000);   
julia> x5 = rand(Bernoulli(0.25), 100) .== 1;
\end{verbatim}
\end{frame}
%=================================================%
\begin{frame}[fragile]
	\frametitle{Statistics with Julia}
	\large
\begin{itemize}
	\item We’ll apply a one sample t-test to x1 and x2. The output below indicates that x2 has a mean which differs significantly from zero while x1 does not. 
	\item This is consistent with our expectations based on the way that these data were generated.
	\item  I’m impressed by the level of detail in the output from \texttt{OneSampleTTest()}: different aspects of the test are neatly broken down into sections (population, test summary and details) and there is automated high level interpretation of the test results.
\end{itemize}

\end{frame}
%=================================================%
\begin{frame}[fragile]
	\frametitle{Statistics with Julia}
	\large
\begin{verbatim}
julia> t1 = OneSampleTTest(x1)
One sample t-test
-----------------
Population details:
parameter of interest:   Mean
value under h_0:         0
point estimate:          -0.013027816861268473
95% confidence interval: (-0.07587776077157478,0.04982212704903784)

Test summary:
outcome with 95% confidence: fail to reject h_0
two-sided p-value:           0.6842692696393744 (not signficant)
\end{verbatim}
\end{frame}
%=================================================%
\begin{frame}[fragile]
	\frametitle{Statistics with Julia}
	\large
\begin{verbatim}
Details:
number of observations:   1000
t-statistic:              -0.40676289562651996
degrees of freedom:       999
empirical standard error: 0.03202803648352013
julia> t2 = OneSampleTTest(x2)
One sample t-test
\end{verbatim}
\end{frame}
%=================================================%
\begin{frame}[fragile]
	\frametitle{Statistics with Julia}
	\large
	\begin{verbatim}
Population details:
parameter of interest:   Mean
value under h_0:         0
point estimate:          0.507852246706
95% confidence interval: (0.4468203,0.568884)

Test summary:
outcome with 95% confidence: reject h_0
two-sided p-value:           2.62561601163e-53 
Details:
number of observations:   1000
t-statistic:              16.32883382693
degrees of freedom:       999
empirical standard error: 0.03110156255427
\end{verbatim}
\end{frame}
%=================================================%
\begin{frame}[fragile]
	\frametitle{Statistics with Julia}
	\large
\begin{itemize}
	\item Using \texttt{pvalue()} we can further interrogate the p-values generated by these tests. 
	\item The values reported in the output above are for the two-sided test, but we can look specifically at values associated with either the left- or right tails of the distribution.
	\item This makes the outcome of the test a lot more specific.
\end{itemize}

\end{frame}
%=================================================%
\begin{frame}[fragile]
	\frametitle{Statistics with Julia}
	\large
	\begin{verbatim}
julia> pvalue(t1)
0.6842692696393744

julia> pvalue(t2)
2.6256160116367554e-53

julia> pvalue(t2, tail = :left)            #
1.0

julia> pvalue(t2, tail = :right)          
1.3128080058183777e-53
\end{verbatim}
\end{frame}
%=================================================%
\begin{frame}[fragile]
	\frametitle{Statistics with Julia}
	\large
The associated confidence intervals are also readily accessible. We can choose between two-sided or left/right one-sided intervals as well as change the significance level.

\begin{verbatim}
julia> ci(t2, tail = :both)               
(0.44682036100064954,0.5688841324132342)

julia> ci(t2, tail = :left)                
(-Inf,0.5590572480083876)

julia> ci(t2, 0.01, tail = :right)         
(0.43538291818831604,Inf)
\end{verbatim}
\end{frame}
%============================================================%
\begin{frame}
\frametitle{Statistics with Julia}
\large
\begin{itemize}
	\item As a second (and final) example we’ll look at \texttt{BinomialTest()}. 
	\item There are various ways to call this function. First, without looking at any particular data, we’ll check whether 25 successes from 100 samples is inconsistent with a 25\% success rate (obviously not and, as a result, we fail to reject this hypothesis).
\end{itemize}

\end{frame}
%============================================================%
\begin{frame}[fragile]
	\frametitle{Statistics with Julia}
	\large
\begin{verbatim}
julia> BinomialTest(25, 100, 0.25)
Binomial test
-------------
Population details:
parameter of interest:   Probability of success
value under h_0:         0.25
point estimate:          0.25
95% confidence interval: (0.16877973809,0.3465524957)

Test summary:
outcome with 95% confidence: fail to reject h_0
two-sided p-value:           1.0 (not signficant)

Details:
number of observations: 100
number of successes:    25
\end{verbatim}
\end{frame}
%============================================================%
\begin{frame}
	\frametitle{Statistics with Julia}
	\large

Next we’ll see whether the Bernoulli samples in x5 provide contradictory evidence to an assumed 50\% success rate (based on the way that x5 was generated we are not surprised to find an infinitesimal p-value and the hypothesis is soundly rejected).

\end{frame}
%=================================================%
\begin{frame}[fragile]
	\begin{verbatim}
julia> BinomialTest(x5, 0.5)
Binomial test
-------------
Population details:
parameter of interest:   Probability of success
value under h_0:         0.5
point estimate:          0.18
95% confidence interval: (0.11031122915326055,0.26947708596681197)

Test summary:
outcome with 95% confidence: reject h_0
two-sided p-value:           6.147806615048005e-11 (extremely significant)

Details:
number of observations: 100
number of successes:    18
\end{verbatim}
\end{frame}
%============================================================%
\begin{frame}
	\frametitle{Statistics with Julia}
	\large
\begin{itemize}
\item There are a number of other tests available in this package, including a range of non-parametric tests which I have not even mentioned above. 
\item Certainly HypothesisTests should cover most of the bases for statistical inference. For more information, read the extensive documentation.
\end{itemize}
\end{frame}

%===========================================================%
\end{document}