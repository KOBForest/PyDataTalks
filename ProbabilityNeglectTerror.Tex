Probability Neglect
2
0
0
Why the government massively overestimates the risks of terrorism.

By John Mueller and Mark G. Stewart
Read more from Slate's Sept. 11 anniversary coverage.

This is adapted from Mueller and Stewart's new Terror, Security, and Money: Balancing the Risks, Benefits, and Costs of Homeland Security. Read yesterday's excerpt  about why the government won't subject homeland security to a cost-benefit analysis, and check back tomorrow for a final article about which—if any—homeland security measures justify their cost.

John Mueller and Mark Stewart. Click image to expand.
John Mueller and Mark Stewart
A recent book by Gregory Treverton, a risk analyst at the RAND Corporation whose work we have found highly valuable, contains a curious reflection:

When I spoke about the terrorist threat, especially in the first years after 2001, I was often asked what people could do to protect their family and home. I usually responded by giving the analyst's answer, what I labeled "the RAND answer." Anyone's probability of being killed by a terrorist today was essentially zero and would be tomorrow, barring a major discontinuity. So, they should do nothing. It is not surprising that the answer was hardly satisfying, and I did not regard it as such.

From this experience, he concluded, "People want information, but the challenge for government is to warn without terrifying."

It is not clear why anyone should find his observation unsatisfying since it simply puts the terrorist threat in general and in personal context, suggesting that excessive alarm about the issue is scarcely called for. It is, one might suspect, exactly the kind of accurate, reassuring, adult, and nonterrifying information people have been yearning for. And it deals frontally with a key issue in risk assessment: evaluating the likelihood of a terrorist attack.

Treverton's "RAND answer," calmly (and accurately) detailing the likelihood of the terrorist hazard and putting it in reasonable context, has scarcely ever been duplicated by politicians and officials in charge of providing public safety. Instead the awkward problem of dealing with exceedingly low probabilities has been finessed—and questionable expenditures accordingly justified—by stratagems that in various ways embrace a form of risk aversion that can be called "probability neglect."

Legal scholar and White House official Cass Sunstein, who seems to have invented the phrase, "probability neglect," assesses the version of the phenomena that comes into being when "emotions are intensely engaged," as they were after 9/11.  Under that circumstance, he argues, "people's attention is focused on the bad outcome itself, and they are inattentive to the fact that it is unlikely to occur." Moreover, they are inclined to "demand a substantial governmental response—even if the magnitude of the risk does not warrant the response."

Playing to this demand, government officials are inclined to focus on worst-case scenarios, presumably in the knowledge, following Sunstein's insight, that this can emotionally justify just about any expenditure no matter how unlikely the prospect the dire event will actually take place.

Analyst Bruce Schneier has written penetratingly of worst-case thinking. He points out that it "involves imagining the worst possible outcome and then acting as if it were a certainty. It substitutes imagination for thinking, speculation for risk analysis, and fear for reason. It fosters powerlessness and vulnerability and magnifies social paralysis. And it makes us more vulnerable to the effects of terrorism."

Another technique is simply to rank relative risk while neglecting to determine the actual magnitude of the risk. It may be true that New York is more likely to be struck by a terrorist than, say, Columbus, Ohio. But it is also more likely to be struck by a tsunami, and not only in Hollywood disaster thrillers. Before spending a lot of money protecting New York from a tsunami, we need to get some sort of sense about what the likelihood of that event actually is, not simply how the risk compares to that borne by other cities. And the same goes for terrorism.

There is also a tendency to inflate the importance of potential terrorist targets. Thus, nearly half of American federal homeland security expenditure is devoted to protecting what the Department of Homeland Security and various presidential and congressional reports and directives rather extravagantly call "critical infrastructure" and "key resources"— assets whose loss would have a "debilitating effect on security, national economic security, public health or safety, or any combination thereof" or are "essential to the minimal operations of the economy or government." It is difficult to imagine what a terrorist group armed with anything less than a massive thermonuclear arsenal could do to hamper such "minimal operations." The terrorist attacks of 9/11 were by far the most damaging in history, yet, even though several major commercial buildings were demolished, both the economy and government continued to function at considerably above the "minimal" level.

A final, and very important, stratagem is to fail to assess, or massively to inflate, the capacities of the terrorists, and therefore by inference both the likelihood they will attack and the consequences of that attack. This is something that should be of absolutely key importance yet, in its big national infrastructure protection report of 2009, the DHS devotes only two paragraphs to describing the nature of the "terrorist adversary." Moreover, none of this fleeting discussion shows any depth, and the report prefers instead to spew out adjectives like "relentless," "patient," and "flexible," terms that scarcely characterize the vast majority of potential terrorists.
