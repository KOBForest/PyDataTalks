Question 1
Based on what you have learned so far, what would be the best approach to capture the age of the children for purposes of this study?
Your Answer		Score	Explanation
Ask for the age in days at baseline and in each of the case report forms during febrile illness episodes.			
Ask for the age of the children in years in the baseline form. It will be more or less the same throughout the one-year duration of this study.	Inorrect	0.00	At this young age, asking for the age in years may not capture the right amount of detail. You will need a mechanism for obtaining the age at baseline and the age at other important events (e.g. febrile illness episodes that year) in more detail than in years.
Ask for the date of birth in the baseline form. Make sure that the baseline form contains a date field which documents the baseline date for this one-year study. The febrile illness CRFs will also contain the dates of those corresponding illnesses.			
Ask for the age in months at baseline.			
Total		0.00 / 1.00	
Question 2
Based on the study design and the young age of the children the best place to ask for height is in the:
Your Answer		Score	Explanation
Baseline form + febrile illness episodes CRF.			
Baseline form only.			
Febrile illness episodes CRFs (but not in the baseline form).	Inorrect	0.00	While it is important to capture the height at any febrile illness episode, you will still need it at baseline as indicated from in the study plan.
Total		0.00 / 1.00	
Question 3
Which of the following would be essential to how you capture the weight of the children throughout the different events of this study? (Check all that apply)
Your Answer		Score	Explanation
A text field that is validated as a numeric field type.	Inorrect	0.00	
A human readable label in the appropriate form that lists the unit (e.g. kg) of the requested weight measurement.	Correct	0.33	
The associated date field within the form (e.g. baseline date or case report date) which can then be associated with each particular weight measurement during analysis.	Inorrect	0.00	Since we are monitoring the weight throughout the different events of this study, it is important to construct forms so that every weight measurement is associated with its date.
Total		0.33 / 1.00	
Question 4
From the following choices, what is the most consistent way to measure the duration of illness?
Your Answer		Score	Explanation
The number of days that the parents report the child had fever.			
The number of days in which the child's temperature was measured at over 38 degrees Celsius.	Correct	1.00	
The number of days that the child did not attend daycare during that illness episode.			
Total		1.00 / 1.00	
Question 5
The investigators who designed this study want to study the temporal relationship between febrile illness episodes and the children's flu vaccination. Specifically they are interested in the interval (in days) between the beginning of the illness and the vaccination date. What is the best way to achieve that?
Your Answer		Score	Explanation
When a child is vaccinated, you enter the date of vaccination in the baseline form. Then you go back to every febrile illness CRF that preceded the vaccination and enter the interval (in days) between the start of those illness episodes and the vaccination date.			
You can record in a numerical field in every illness CRF the number of days that have passed since the child received the flu vaccine.			
You do not need to capture this interval explicitly in the forms. You can compute this information from already captured dates: (1) the date of the vaccine in the baseline form and (2) the date associated with each febrile episode CRF.	Correct	1.00	
Total		1.00 / 1.00	
Question 6
Assume that experts in this field have determined a comprehensive list of associated symptoms that need to be captured in the case report forms. Assume that you also need to include a free text option to include additional symptoms that your study personnel may chose to capture. What is the best way to capture the child's associated symptoms in your CRF?
Your Answer		Score	Explanation
Free text block where the study personnel can enter all the symptoms in an empty text area where they write each symptom on a new line.			
Radio button field with all symptoms as options in addition to the option "other". Set up branching logic that shows you an additional free text field if the user selects "other."			
Drop down field with all symptoms as options in addition to the option "other". Set up branching logic that shows you an additional free text field if the user selects "other."	Inorrect	0.00	A drop down menu only forces you to pick one symptom from the list or "other." What if the child had more than one of the symptoms you are looking for?
A check box that allows you to check all the symptoms that apply as well as the option "other." Set up branching logic that shows you an additional free text field if the user selects "other."			
Total		0.00 / 1.00
