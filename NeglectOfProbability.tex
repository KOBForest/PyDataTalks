Neglect of probability
From Wikipedia, the free encyclopedia
The neglect of probability, a type of cognitive bias, is the tendency to completely disregard probability when making a decision under uncertainty and is one simple way in which people regularly violate the normative rules for decision making. Small risks are typically either neglected entirely or hugely overrated, the continuum between the extremes is ignored. The term probability neglect was coined by Cass Sunstein.[1]
There are many related ways in which people violate the normative rules of decision making with regard to probability including the hindsight bias, the neglect of prior base rates effect, and the gambler's fallacy. This bias, though, is notably different from the preceding biases because with this bias, the actor completely disregards probability when deciding, instead of incorrectly using probability, as the actor does in the above examples.
Baron, Granato, Spranca, and Teubal (1993) studied the bias. They did so by asking children the following question:
Susan and Jennifer are arguing about whether they should wear seat belts when they ride in a car. Susan says that you should. Jennifer says you shouldn't... Jennifer says that she heard of an accident where a car fell into a lake and a woman was kept from getting out in time because of wearing her seat belt, and another accident where a seat belt kept someone from getting out of the car in time when there was a fire. What do you think about this?
Jonathan Baron (2000) notes that subject X responded in the following manner:
A: Well, in that case I don't think you should wear a seat belt.
Q (interviewer): How do you know when that's gonna happen?
A: Like, just hope it doesn't!
Q: So, should you or shouldn't you wear seat belts?
A: Well, tell-you-the-truth we should wear seat belts.
Q: How come?
A: Just in case of an accident. You won't get hurt as much as you will if you didn't wear a seat belt.
Q: OK, well what about these kinds of things, when people get trapped?
A: I don't think you should, in that case.
It is clear that subject X completely disregards the probability of an accident happening versus the probability of getting hurt by the seat belt in making the decision. A normative model for this decision would advise the use of expected-utility theory to decide which option would likely maximize utility. This would involve weighing the changes in utility in each option by the probability that each option will occur, something that subject X ignores.
Another subject responded to the same question:
A: If you have a long trip, you wear seat belts half way.
Q: Which is more likely?
A: That you'll go flyin' through the windshield.
Q: Doesn't that mean you should wear them all the time?
A: No, it doesn't mean that.
Q: How do you know if you're gonna have one kind of accident or the other?
A: You don't know. You just hope and pray that you don't.
Here again, the subject disregards the probability in making the decision by treating each possible outcome as equal in his reasoning.
Baron (2000) suggests that adults may suffer from the bias as well, especially when it comes to difficult decisions like a medical decision under uncertainty. This bias could make actors drastically violate expected-utility theory in their decision making, especially when a decision must be made in which one possible outcome has a much lower or higher utility but a small probability of occurring (e.g. in medical or gambling situations). In this aspect, the neglect of probability bias is similar to the neglect of prior base rates effect.
In another example of near-total neglect of probability, Rottenstreich and Hsee (2001) found that the typical subject was willing to pay $10 to avoid a 99% chance of a painful electric shock, and $7 to avoid a 1% chance of the same shock. (They suggest that probability is more likely to be neglected when the outcomes are emotion arousing.)
